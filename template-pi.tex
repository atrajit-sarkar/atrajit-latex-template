% ============================================================================
% SAMPLE DOCUMENT USING ATRAJIT-PI THEME
% ============================================================================
% To use this theme, simply replace \usepackage{atrajit} with:
% \usepackage{atrajit-pi}
% ============================================================================

\documentclass[11pt, a4paper]{article}

% ============================================================================
% PACKAGE SELECTION - Choose Your Theme
% ============================================================================

% Original Zeta Theme (Blue colors, zeta symbols)
% \usepackage{atrajit}

% NEW: Elegant Pi Theme (Warm brown/gold colors, pi symbols)
\usepackage{atrajit-pi}

% ============================================================================
% DOCUMENT METADATA
% ============================================================================

\title{\textbf{Elegant Mathematics with the Pi Theme}\\
       \large A Demonstration of the Atrajit-Pi Academic Package}
\author{Atrajit Sarkar}
\date{\today}

% Bibliography file (if needed)
% \addbibresource{references.bib}

% ============================================================================
% DOCUMENT CONTENT
% ============================================================================

\begin{document}

\maketitle

\begin{abstract}
This document demonstrates the elegant \texttt{atrajit-pi} theme, featuring warm academic colors, sophisticated typography, and pi-themed decorations. The package provides a complete professional look for mathematical documents, research papers, and academic notes.
\end{abstract}

\tableofcontents
\newpage

% ============================================================================
\section{Introduction to the Pi Theme}
% ============================================================================

The \texttt{atrajit-pi} package provides an elegant alternative to the original \texttt{atrajit} theme. Key features include:

\begin{itemize}
    \item Warm color palette (browns and golds) for sophisticated appearance
    \item Pi-themed decorative elements throughout
    \item Hexagonal and circular badge designs
    \item Enhanced theorem environments with colored frames
    \item Elegant ornamental headers and footers
\end{itemize}

\subsection{Color Scheme}

The theme uses a carefully selected academic color palette:
\begin{itemize}
    \item \textcolor{piPrimary}{\textbf{Primary:}} Saddle Brown for main elements
    \item \textcolor{piSecondary}{\textbf{Secondary:}} Dark Goldenrod for accents
    \item \textcolor{piAccent}{\textbf{Accent:}} Peru for highlights
    \item \textcolor{piLight}{\textbf{Background:}} Wheat for subtle fills
\end{itemize}

% ============================================================================
\section{Mathematical Environments}
% ============================================================================

\subsection{Theorem Environments}

The pi theme includes beautifully styled theorem environments:

\begin{theorem}[Fundamental Theorem of Calculus]
Let $f$ be continuous on $[a,b]$. Then
\[
\int_a^b f(x)\,dx = F(b) - F(a)
\]
where $F$ is any antiderivative of $f$.
\end{theorem}

\begin{proof}
This follows from the definition of the definite integral and the chain rule.
\end{proof}

\begin{lemma}
Every bounded sequence in $\RR^n$ has a convergent subsequence.
\end{lemma}

\begin{proposition}
The set $\QQ$ of rational numbers is countable.
\end{proposition}

\begin{corollary}
The set of algebraic numbers is also countable.
\end{corollary}

\subsection{Definitions and Examples}

\begin{definition}[Metric Space]
A \defterm{metric space} is a pair $(X, d)$ where $X$ is a set and $d: X \times X \to \RR$ satisfies:
\begin{enumerate}
    \item $d(x,y) \geq 0$ with equality iff $x = y$
    \item $d(x,y) = d(y,x)$ (symmetry)
    \item $d(x,z) \leq d(x,y) + d(y,z)$ (triangle inequality)
\end{enumerate}
\end{definition}

\begin{example}
The Euclidean space $\RR^n$ with the standard metric $d(x,y) = \|x-y\|$ forms a metric space.
\end{example}

\begin{remark}
Not every topological space can be metrized.
\end{remark}

% ============================================================================
\section{Special Features}
% ============================================================================

\subsection{Custom Markers}

The pi theme includes several custom markers:

\todostar\ This is a TODO item that needs attention.

\notebell\ This is an important note to remember.

\importantmark\ This highlights critical information.

\subsection{Text Highlighting}

You can emphasize mathematical terms using \verb|\mathterm{term}|: The \mathterm{eigenvalue} is a fundamental concept.

For definitions, use \verb|\defterm{term}|: A \defterm{manifold} is a topological space.

\subsection{Key Concept Box}

Use the \verb|\keybox| command for important concepts:

\keybox{%
\textbf{Key Insight:} The relationship between $\pi$ and $e$ is given by Euler's identity:
\[
e^{i\pi} + 1 = 0
\]
This is considered one of the most beautiful equations in mathematics.
}

% ============================================================================
\section{Mathematical Examples}
% ============================================================================

\subsection{Important Constants}

The constant \piconst\ appears throughout mathematics:

\begin{equation}
\int_{-\infty}^{\infty} e^{-x^2}\,dx = \sqrt{\pi}
\end{equation}

\begin{equation}
\zeta(2) = \sum_{n=1}^{\infty} \frac{1}{n^2} = \frac{\pi^2}{6}
\end{equation}

\subsection{Complex Analysis}

\begin{theorem}[Residue Theorem]
Let $f$ be analytic inside and on a simple closed contour $C$ except for isolated singularities. Then:
\[
\oint_C f(z)\,dz = 2\pi i \sum \text{Res}(f, z_k)
\]
\end{theorem}

% ============================================================================
\section{Conclusion}
% ============================================================================

The \texttt{atrajit-pi} theme provides a sophisticated, elegant alternative for academic documents. Simply change your package declaration from \verb|\usepackage{atrajit}| to \verb|\usepackage{atrajit-pi}| to transform your document's appearance.

\sectionbreak

\begin{center}
\textit{Happy typesetting with the Pi theme!}
\end{center}

% ============================================================================
% BIBLIOGRAPHY (if needed)
% ============================================================================
% \printbibliography

\end{document}
