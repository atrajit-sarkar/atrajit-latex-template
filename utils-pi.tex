%=============================================================================
% UTILITY MACROS AND VISUAL ELEMENTS - PI THEME
%=============================================================================

%-----------------------------------------------------------------------------
% Mathematical Decorations - Pi Symbol Design
%-----------------------------------------------------------------------------

% Elegant pi symbol with sophisticated hexagonal frame
\providecommand{\zetacircle}{%
    \tikz[baseline=-0.6ex]{
        % Outer hexagonal frame (elegant geometry)
        \draw[
            piDark,
            line width=0.6pt,
            rounded corners=0.5pt
        ] (0,0.55em) -- (0.48em,0.275em) -- (0.48em,-0.275em) -- 
          (0,-0.55em) -- (-0.48em,-0.275em) -- (-0.48em,0.275em) -- cycle;
        
        % Inner circle with warm gradient effect
        \fill[piLight] (0,0) circle (0.48em);
        \draw[
            piSecondary,
            line width=0.5pt
        ] (0,0) circle (0.48em);
        
        % Pi symbol in elegant typography
        \node[
            text=piPrimary,
            font=\normalsize\boldmath,
            inner sep=0pt
        ] at (0,0) {$\pi$};
    }%
}

% Alternative: Circular pi badge with ornamental border
\providecommand{\pibadge}{%
    \tikz[baseline=-0.6ex]{
        % Ornamental outer ring
        \foreach \angle in {0,45,90,135,180,225,270,315} {
            \fill[piSecondary, opacity=0.3] 
                (\angle:0.62em) circle (0.08em);
        }
        
        % Main circular frame
        \fill[piLight] (0,0) circle (0.52em);
        \draw[piPrimary, line width=0.6pt] (0,0) circle (0.52em);
        \draw[piSecondary, line width=0.3pt, opacity=0.5] (0,0) circle (0.46em);
        
        % Pi symbol
        \node[
            text=piDark,
            font=\normalsize\boldmath,
            inner sep=0pt
        ] at (0,0) {$\pi$};
    }%
}

%-----------------------------------------------------------------------------
% Footer Design: Elegant Ornamental Banner
%-----------------------------------------------------------------------------

\providecommand{\fancyFootPageNum}{
    \setlength{\footskip}{60pt}
    \renewcommand{\footrulewidth}{0pt}
    \fancyfoot[C]{
        \vspace{6pt}
        \begin{tikzpicture}[baseline=(page.base)]
            % Decorative left flourish
            \node[
                text=piSecondary,
                font=\footnotesize
            ] (left) at (-1.5,0) {$\diamond$};
            
            % Ornamental line left
            \draw[piAccent, line width=0.4pt] 
                (-1.3,0) -- (-0.5,0);
            
            % Main page number box with elegant frame
            \node[
                rectangle,
                draw=piPrimary,
                line width=0.6pt,
                fill=piLight,
                inner xsep=8pt,
                inner ysep=3pt,
                rounded corners=2pt
            ] (page) at (0,0) {%
                \footnotesize\bfseries\color{piDark}%
                \thepage%
            };
            
            % Pi symbol decoration
            \node[
                text=piSecondary,
                font=\scriptsize
            ] at (0, -0.45) {\zetacircle};
            
            % Ornamental line right
            \draw[piAccent, line width=0.4pt] 
                (0.5,0) -- (1.3,0);
            
            % Decorative right flourish
            \node[
                text=piSecondary,
                font=\footnotesize
            ] (right) at (1.5,0) {$\diamond$};
            
        \end{tikzpicture}
    }
}

%-----------------------------------------------------------------------------
% Header Design: Sophisticated Academic Layout
%-----------------------------------------------------------------------------

\providecommand{\fancyHeaderContent}{
    % Double header rule with elegant styling
    \renewcommand{\headrulewidth}{0pt}
    
    % Custom header rule with decorative elements
    \fancyhead[L]{%
        \begin{tikzpicture}[remember picture, overlay]
            \draw[piAccent, line width=0.4pt] 
                ([yshift=-2pt]current page.north west) ++(\oddsidemargin+1in,0) -- 
                ++(\textwidth,0);
            \draw[piSecondary, line width=0.2pt, opacity=0.5] 
                ([yshift=-3.5pt]current page.north west) ++(\oddsidemargin+1in,0) -- 
                ++(\textwidth,0);
        \end{tikzpicture}%
    }
    
    % Center header: Section information with pi decoration
    \fancyhead[C]{%
        \footnotesize\itshape\color{piDark}%
        \leftmark%
        \enspace%
        \textcolor{piAccent}{$\sim$}%
        \;\zetacircle\;%
        \textcolor{piAccent}{$\sim$}%
        \enspace%
        \rightmark%
    }
    
    % Right header: Elegant graduation cap icon
    \fancyhead[R]{%
        \footnotesize\textcolor{piSecondary}{\faGraduationCap}%
    }
}

%-----------------------------------------------------------------------------
% Ornamental Section Break
%-----------------------------------------------------------------------------

% Decorative section separator
\providecommand{\sectionbreak}{%
    \vspace{0.5em}
    \begin{center}
        \textcolor{piSecondary}{$\diamond$}%
        \enspace%
        \textcolor{piAccent}{\zetacircle}%
        \enspace%
        \textcolor{piSecondary}{$\diamond$}%
    \end{center}
    \vspace{0.5em}
}
