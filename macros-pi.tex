%=============================================================================
% CUSTOM MACROS AND DOCUMENT SETTINGS - PI THEME
%=============================================================================

%-----------------------------------------------------------------------------
% Mathematical Constants and Notation
%-----------------------------------------------------------------------------

% Pi with proper spacing
\newcommand{\piconst}{\ensuremath{\pi}}

% Common mathematical sets with pi theme colors
\newcommand{\RR}{\ensuremath{\mathbb{R}}}
\newcommand{\NN}{\ensuremath{\mathbb{N}}}
\newcommand{\ZZ}{\ensuremath{\mathbb{Z}}}
\newcommand{\QQ}{\ensuremath{\mathbb{Q}}}
\newcommand{\CC}{\ensuremath{\mathbb{C}}}

%-----------------------------------------------------------------------------
% Equation Numbering Configuration
%-----------------------------------------------------------------------------

% Number equations by section (elegant academic style)
\numberwithin{equation}{section}

% Optional: Number equations by subsection
\newcommand{\useSubsectionEqNum}{
  \makeatletter
  \numberwithin{equation}{subsection}
  \makeatother
}

%-----------------------------------------------------------------------------
% Footnote Styling - Pi Theme
%-----------------------------------------------------------------------------

% Custom footnote format using pi notation: π₁, π₂, etc.
\makeatletter
\renewcommand{\thefootnote}{$\pi_{\arabic{footnote}}$}
\makeatother

%-----------------------------------------------------------------------------
% Custom List Environments
%-----------------------------------------------------------------------------

% Elegant enumerate with custom colors
\setlist[enumerate]{
    label=\textcolor{piPrimary}{\arabic*.},
    leftmargin=*,
    itemsep=0.3em
}

% Elegant itemize with custom symbols
\setlist[itemize]{
    label=\textcolor{piSecondary}{$\triangleright$},
    leftmargin=*,
    itemsep=0.3em
}

%-----------------------------------------------------------------------------
% Text Highlighting Macros
%-----------------------------------------------------------------------------

% Highlight important mathematical terms
\newcommand{\mathterm}[1]{\textcolor{piPrimary}{\textbf{#1}}}

% Emphasize definitions
\newcommand{\defterm}[1]{\textcolor{piSecondary}{\textit{#1}}}

% Inline code or notation
\newcommand{\code}[1]{\texttt{\textcolor{piDark}{#1}}}

%-----------------------------------------------------------------------------
% Box Environments for Special Content
%-----------------------------------------------------------------------------

% Key concept box
\newcommand{\keybox}[1]{%
    \begin{center}
        \begin{tikzpicture}
            \node[
                rectangle,
                draw=piPrimary,
                line width=0.8pt,
                fill=piLight,
                inner sep=10pt,
                rounded corners=3pt,
                text width=0.9\textwidth
            ] {%
                \textcolor{piDark}{#1}%
            };
        \end{tikzpicture}
    \end{center}
}

%-----------------------------------------------------------------------------
% Reference Shortcuts
%-----------------------------------------------------------------------------

% Elegant theorem references
\newcommand{\thmref}[1]{Theorem~\ref{#1}}
\newcommand{\lemref}[1]{Lemma~\ref{#1}}
\newcommand{\propref}[1]{Proposition~\ref{#1}}
\newcommand{\corref}[1]{Corollary~\ref{#1}}
\newcommand{\defref}[1]{Definition~\ref{#1}}
\newcommand{\exref}[1]{Example~\ref{#1}}

%-----------------------------------------------------------------------------
% Chapter/Section Numbering Style
%-----------------------------------------------------------------------------

% Optional: Use Roman numerals for sections
\newcommand{\useRomanSections}{
    \renewcommand{\thesection}{\Roman{section}}
}
